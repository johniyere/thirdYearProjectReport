\chapter{Conclusion}
In conclusion, the project covers a wide breadth of topic areas in the approach taking to solving the problems highlighted in \autoref{chap:Intro}. The project is aimed at trail runners who need a platform to be able to create, share and discover trails.

The interface discussed in \autoref{chap:TrailInterface} explains how we provide a well informative interface allowing users to create this interface. The map provider used is industry standard and sufficient research and comparisons was done to to come to this decision.

By examining the strengths and weaknesses of different techniques to creating a Recommender system, the project not only justifies the Recommender system approach with a detailed explanation on the benefits of the techniques used but also delivers a system that is able to compete with other proposed solutions. 

Thee project includes a full stack development process, bearing the results of all parts of the project in the form of a very presentable, polished and feature rich dynamic web application.

Finally, many parts of the project where evaluated to ensure that the systems where working efficiently. This included use of manual testing from clever development tools, industry testing using standing test approach methods and statistical analysis from empirical evaluation metrics. 

Overall, I believe that the project explores many research areas and produces a deliverable that has been well executed, well-evaluated and that I am proud of.